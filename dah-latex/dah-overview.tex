\chapter{DAH: Course Overview}
\label{sec:overview}

\section{Introduction}

Data Acquisition and Handling (DAH) is a  Senior Honours course, which was introduced in 2014 during a review of whole degree programme. DAH will introduce you to methods and tools of modern Data Acquisition and Handling, including Analog and Digital electronics, reading out sensors (detectors), handling and interpreting data. This course replaces JH Electronics Methods. Note that some parts of Electronics Methods (digital and analog electronics) are now taught in 2nd year Practical Physics. The DAH course will focus on data acquisition and data analysis.


\section{Schedule}
%In weeks 1  and 2 (Tuesday 19/26 and Thursday 21/28 September 2017)
In week 1 (Tuesday 17 and Thursday 19 September 2019)
there will be lectures %and a python tutorial session 
introducing the DAH course.
Laboratory work commences in week 1 (Tuesday 17 September 2019) and finishes at the end of week 11 (Thursday 28 November 2019). The laboratory sessions will take place on Tuesday and Thursday afternoons from 14:00 to 17:00 and you need to attend one of these sessions. Timetabling will automatically allocate one of the afternoons for you. The laboratory sessions will be held in JCMB 3301.
\begin{itemize}
\item Tuesday 17 September, 2 to 3 pm,  Lecture Theatre 100, Joseph Black Building \\Introduction to course and Lecture 1.
\item Thursday 19 September, 2 to 3 pm, Lecture Theatre 2, Hudson Beare Building : Lecture 2.
%\item Tuesday 26 September,  2 to 3 pm, JCMB Lecture Theatre LTC: Lecture 3.
%\item Thursday 28 September,  2 to 3 pm, JCMB Lecture Theatre LTC: Lecture 4.
\item week 1, Tuesday 17 or Thursday  19 September, 3 to 5 pm, JCMB room 3301: \\laboratory work.
\item weeks 2 to 11:  Tuesday or Thursday 2 to 5 pm, JCMB room 3301: \\laboratory work.
\end{itemize}



\section{Syllabus}

The outline syllabus is as follows:\\
\begin{itemize}
\item Analogue signal processing. Treatment of noise. Filtering. Buffering using sample and hold;
\item Analogue to digital conversion. Sampling rates. Characteristics \& errors; 
\item Digital to analogue signal conversion;
\item Communication protocols (Bus standards). Input/Output (I/O);
\item Digital signal processing. Triggering. Fourier transforms;
\item Data acquisition using a Raspberry Pi and Arduino and python;
\item Advanced data analysis. Multi-parameter likelihood fits;
\item Practical examples, e.g. temperature sensors, ultrasound sensors, FFT spectrum analyser,
synthesizer, digital signal generators, motion sensors, remote sensing, image processing, with CCDs.
\end{itemize}

\section{Learning Outcome}
 On completion of this course, the student will be able to:

\begin{enumerate}
\item    Understand core concepts of data acquisition, data handling and data analysis in physical sciences;
\item    Apply standard practical laboratory techniques (e.g. routine handling of data acquisition equipment and writing short, procedural computer programs) as directed in a script to achieve a stated goal;
\item    Apply advanced practical laboratory techniques (e.g. handling of complex data acquisition equipment, and writing data acquisition computer programs) with limited direction to achieve a stated or open-ended goal;
\item    Apply advanced data handling and data analysis techniques (e.g., data selection and representation, multi-parameter likelihood fits and writing data analysis computer programs) with limited direction to achieve a stated or open-ended goal;
\item    Present a record of an experiment or computation in an appropriate, clear and logical written form (e.g. laboratory notebook, laboratory report, fully documented computer code), augmented with figures graphs, audio or movies where appropriate.
\end{enumerate}

\section{Laboratory work}

The laboratory sessions of the DAH course will take place in JCMB 3301 on Tuesday and Thursday afternoons from 14:00 to 17:00 (15:00 - 17:00 in week 1). You will need to attend one of these sessions. Timetabling will automatically allocate either Tuesday or Thursday for you. In the laboratory you will work in pairs, so you will need to choose a partner.

The laboratory work consists of checkpoints and projects, which are described in detail in Chapters~\ref{sec:checkpoints} and~\ref{sec:projects}.  You will work with a Raspberry-Pi, which is a credit-card sized computer that plugs into a computer/TV screen and a keyboard. A manual will be provided. To control the Raspberry-Pi you will need to write python code. Example python scripts and code snippets will be provided on github, see \url{https://github.com/fmuheim/DAH}. 

During weeks 1 to 6 of the DAH course you will need to complete six checkpoints. In each checkpoint you will learn a specific aspect of data acquisition or data handling and complete a prescribed number of tasks. You will work in pairs during the checkpoints.
Each pair will have their own set of kit, however the Raspberry-Pi's will be shared between the Tuesday and Thursday afternoon sessions. Each pair will be given a yellow box with the required kit in which you can preserve your work for use in the following week. The boxes should be labelled with your names.

You should maintain a clear record of your work in a laboratory notebook as you work through the checkpoints. This must include diagrams of the circuits used. Python code written on the Raspberry-Pi must include explanatory comments and at each check point you need to demonstrate that your code compiles and runs correctly.  Each partner will need to maintain their own notebook to demonstrate that a checkpoint has been completed. 



In weeks 7 to 11 of the DAH course, you will carry out a project during the laboratory sessions. You will continue to attend during the same afternoon as for the checkpoints.
The projects will build upon what you have learned during the checkpoints, but you will also encounter new material. While the checkpoints concentrated on specific data acquisition techniques and data handling methods the projects will allow you to progress towards building a small DAQ and/or data analysis system. The projects will have an open-ended aspect and are an opportunity where you can show your own initiative and demonstrate your experimental and computational skills. 

You will be able to choose from a list of projects, but due to the availability of equipment, the number of spaces for each project will be limited. 
A signup sheet will be provided.

The equipment specific to each DAH project will be available in red boxes. Some of the parts, including Arduinos and loudspeakers, as well as the Raspberry-Pi's, will be shared between Tuesday and Thursday afternoon sessions. In addition, each pair will continue to use the yellow box in which you can preserve your work for use in the following week. 

For the DAH project you will continue to work in pairs. Throughout the project, each of you should maintain a clear record of your work in your laboratory notebook. As an example, diagrams of built circuits must be included. Python code written on the Raspberry-Pi must include explanatory comments.  Each partner will be required to submit an individual report for the project. 
\vfill
\newpage

\section{Assessment} 

Data Acquisition and Handling is a continuously assessed course. The overall DAH  assessment will be made from three parts. The sum of the marks achieved while carrying out the checkpoints will count for 30\% of the total course mark.  The marks obtained for the DAH project will count for 60\% of the total course mark. In addition, there will be a quiz/hand-in which will count for 10\% of the total course mark.

\subsection{Assessment of checkpoints}

There are six checkpoints and checkpoints 2 to 6 will be assessed by one of the demonstrators during the laboratory hours. 
The assessment will be performed when you decide that you have completed the tasks for a checkpoint or parts thereof as allowed by the marking scheme. For each checkpoint a total score of between 8 and 10 marks will be awarded.  In total up to 45 marks will be available for the checkpoints.
Please note the following. 
\begin{itemize}

\item While working in pairs, each student will be assessed separately. The marks awarded need not to be the same for both students. 

\item A marking scheme will be provided. Some of the tasks of each checkpoint may be marked together.

\item You will only be awarded marks if you can demonstrate that the relevant circuit functions correctly and that your python code achieves the requested results. 

\item You will be assessed on your (individual) laboratory notebook and  the structure and readability of your (joint) python code.

\item You will be required to answer the questions given in the checkpoints.

\item  You are required to have separate python scripts for each task of a checkpoint. By doing this you will be able  to demonstrate all parts of a checkpoint and not just the last task. For some tasks you will need to write results or figures to a file. Make sure that these are not overwritten by the next task.

\item You should be able to complete each checkpoint  in one afternoon. The deadlines for assessing checkpoints are as follows:\\
%
%\begin{itemize}
\\The deadline for checkpoint 2 is week 3,  for checkpoint 3 it is week 4, for checkpoint 4 it is week 5,  for checkpoint 5 it is week 6 and for checkpoint 6 it is week 7.
\end{itemize}
The overall laboratory assessment will be made from the sum of marks of the check points, which constitutes 30\% of the total course mark.

\subsection{Project assessment}

Your DAH project will be assessed through the submitted material. This includes your project report and your DAH software (e.g. Python scripts) and, if deemed useful, supplementary material. Guidance on how to prepare these items is given below. The project will be marked according to the University Common Marking Scheme. 

\subsubsection{Report Preparation}

For how to write a proper report we refer you to the workshop slides on report writing in the Senior Honours  (SH) Projects course, which are available at
\url{https://www.wiki.ed.ac.uk/display/SP/SH+Projects}.
The basic layout of a DAH report will be similar to an SH project report with the main difference being that a good DAH report will be shorter and should typically be approximately 7  pages long. It is expected that the report is typed. Most students use LaTeX or Word, either is fine.

When planning and writing a report, you need to be selective about what to include in your report, it should be a concise technical document. However, it also needs to contain all the information required for the reader to understand what was achieved, i.e. with your report you need to be able to demonstrate to what extend the project was carried out successfully. It is often useful to include circuit diagrams, pictures of the setup, or plots of measurements. A good report would allow a fellow student to be able to reproduce your work. Students are advised to start writing their report as the project progresses. Experience shows that report writing  will take longer than anticipated.

Each partner is required to submit an individual report for the project. If you are working with a partner, this report must make it clear which parts of the project were carried out together, which parts are only your work, and which parts were only carried out by your partner. The report must contain a signed declaration, which will be available on Learn.

\subsubsection{Programming Code: e.g. Python}

Programming code for the DAH project, e.g. using Python, written on the Raspberry-Pi or on another computer, must include explanatory comments. When reading a (python) script, a reader should easily be able to understand what the script will do. All code written (in python or another programming language)  for the DAH project will need to be submitted using the "Assessment" tool on Learn.  The files should be bundled up in a .zip or .tar file. A README file should be included. Submission details will be provided.

\subsubsection{Supplementary Material}

You are encouraged to submit supplementary material if you consider the material as a part of the project that does not fit into the report format. This could include your laboratory notebook, output files produced by running a python script, short videos that demonstrate how your project works, ... These can be submitted  using the "Assessment" tool on Learn, or separately on a USB drive or a link to a webpage. If you have questions about the suitability of material, please consult with the Course Organiser. All such supplementary material must be clearly listed in an appendix to the report and referred to in the main text. 

\subsubsection{Submission Deadline:}
The assessed material for the DAH projects will need to be submitted by {\bf 12.00 NOON on Friday, 29th November 2019}. By the deadline you must have submitted 
\begin{itemize}
\item an electronic version of your project report to Turnitin via Learn.
At the beginning of your submission you will need to declare an "Own Work Declaration";
\item supplementary material via Learn. If required (usually not the case), 
you can submit supplementary material to the Teaching Office in JCMB (Room 4309);
\item and any supplementary material. 
\end{itemize}
The marks obtained for the DAH project will count for 60\% of the total course mark. 
Reports submitted after the deadline will receive a penalty of 5\% (equivalent to 3 marks out of 60) for each calendar day by which the deadline is exceeded. Students who, for good reason, find they are unable to meet the deadline, should contact the DAH Project Organiser and Course Secretary before
the deadline.

\subsection{Quiz}
Midway during the course, you will be need to submit a quiz/hand-in on  questions about data acquisition and handling material. You will be given two weeks to solve these questions on your own time, i.e. you should not use laboratory hours to solve the quiz questions. The exact deadline for handing in the quiz will be announced on Learn, it will be around the end of week 7 of the semester. 
The quiz will count for 10\% of the total course mark. 

\section{Plagiarism:}
The University regulations on plagiarism apply, see Section 27 of the Taught Assessment Regulations, available online at \url{http://www.ed.ac.uk/schools-departments/academic-services/policies-regulations/regulations/assessment}.


