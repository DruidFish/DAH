\chapter{DAH: Course Overview}
\label{sec:overview}

\section{Introduction}

Data Acquisition and Handling (DAH) is a Senior Honours course, which was introduced in 2014 during a review of the whole degree programme.
DAH will introduce you to methods and tools of modern Data Acquisition and Handling, including analogue and digital electronics, reading out sensors (detectors), handling and interpreting data.
This course replaces JH Electronics Methods.
Note that some parts of Electronics Methods (digital and analogue electronics) are now taught in 2nd year Practical Physics.
The DAH course will focus on data acquisition and data analysis.

\section{Schedule}

%In weeks 1  and 2 (Tuesday 19/26 and Thursday 21/28 September 2017)
%In week 1 (Tuesday 22 and Thursday 24 September 2020)
%there will be lectures %and a python tutorial session 
%introducing the DAH course.
Laboratory work commences in week 1 (Tuesday 22 September 2020)
%and finishes at the end of week 11 (Thursday 28 November 2019).
and finishes at the end of week 6 (Thursday 29 October 2020).
The laboratory sessions will take place on Tuesday and Thursday afternoons from 14:00 to 17:00 --- you need to attend one of these sessions per week.
Timetabling will automatically allocate one of the afternoons for you.
The laboratory sessions will be held in JCMB 3201, 3301, 3307 and 3208.
Please go to your allocated room, and use the same bench space throughout the course.
On the course Learn page you will find Risk Assessment information regarding use of the labs: please read this and use the electronic form to indicate that you have done so.

Those unable to attend laboratory sessions in person can arrange remote access, working with a partner inside the lab.
Please email the course administrator to organise this.
%\begin{itemize}
%\item Tuesday 17 September, 2 to 3 pm,  Lecture Theatre 100, Joseph Black Building \\Introduction to course and Lecture 1.
%\item Thursday 19 September, 2 to 3 pm, Lecture Theatre 2, Hudson Beare Building : Lecture 2.
%%\item Tuesday 26 September,  2 to 3 pm, JCMB Lecture Theatre LTC: Lecture 3.
%%\item Thursday 28 September,  2 to 3 pm, JCMB Lecture Theatre LTC: Lecture 4.
%\item week 1, Tuesday 17 or Thursday  19 September, 3 to 5 pm, JCMB room 3301: \\laboratory work.
%\item weeks 2 to 11:  Tuesday or Thursday 2 to 5 pm, JCMB room 3301: \\laboratory work.
%\end{itemize}

After all laboratory work is finished (week 7 onwards) there will be a data-analysis project that will be performed outside of the laboratory.

\newpage
\section{Syllabus}

The outline syllabus is as follows:
\begin{itemize}
\item Analogue signal processing. %Treatment of noise. Filtering. Buffering using sample and hold;
\item Analogue to digital conversion; sampling rates; characteristics \& errors
\item Digital to analogue signal conversion
\item Communication protocols (Bus standards) and general Input/Output (I/O)
%\item Digital signal processing. Triggering. Fourier transforms
\item Computer data acquisition using a Raspberry Pi and Python %Arduino and python;
\item Advanced data analysis; multi-parameter likelihood fits
%\item Practical examples, e.g. temperature sensors, ultrasound sensors, FFT spectrum analyser, synthesizer, digital signal generators, motion sensors, remote sensing, image processing, with CCDs.
\item Practical examples, e.g. temperature sensors, digital signal generators, light sensors
\end{itemize}

\section{Learning Outcomes}

On completion of this course, the student will be able to:
\begin{enumerate}
\item Understand core concepts of data acquisition, data handling and data analysis in physical sciences
\item Apply standard practical laboratory techniques (e.g. routine handling of data acquisition equipment and writing short, procedural computer programs) as directed in a script to achieve a stated goal
\item Apply advanced practical laboratory techniques (e.g. handling of complex data acquisition equipment and writing data acquisition computer programs) with limited direction to achieve a stated or open-ended goal
\item Apply advanced data handling and data analysis techniques (e.g., data selection and representation, multi-parameter likelihood fits and writing data analysis computer programs) with limited direction to achieve a stated or open-ended goal
\item Present a record of an experiment or computation in an appropriate, clear and logical written form (e.g. laboratory notebook, laboratory report, fully documented computer code), augmented with figures, graphs, audio, or movies where appropriate
\end{enumerate}

\newpage
\section{Laboratory work}

The laboratory sessions of the DAH course will take place in JCMB 3301 (and nearby rooms) on Tuesday and Thursday afternoons from 14:00 to 17:00.% (15:00 - 17:00 in week 1).
You will need to attend one of these sessions.
Timetabling will automatically allocate either Tuesday or Thursday for you, and a room number.
%In the laboratory you will work in pairs, so you will need to choose a partner.

%The laboratory work consists of checkpoints and projects, which are described in detail in Chapters~\ref{sec:checkpoints} and~\ref{sec:projects}.
The laboratory work consists of checkpoints, which are described in detail in Chapter~\ref{sec:checkpoints}.
You will work with a Raspberry Pi, which is a credit-card sized computer that plugs into a computer/TV screen and a keyboard.
A manual will be provided.
To control the Raspberry Pi you will need to write Python code.
Example Python scripts and code snippets will be provided in this document and on github, see \url{https://github.com/fmuheim/DAH}. 

During weeks 1 to 6 of the DAH course you will need to complete five checkpoints. %six checkpoints.
In each checkpoint you will learn a specific aspect of data acquisition or data handling and complete a prescribed set of tasks.
%You will work in pairs during the checkpoints.
%Each pair will have their own set of kit, however the Raspberry Pi's will be shared between the Tuesday and Thursday afternoon sessions.
Every student will have their own set of kit, however the Raspberry Pi's will be shared between the Tuesday and Thursday afternoon sessions.
%Each pair will be given a yellow box with the required kit in which you can preserve your work for use in the following week.
Each student will be given a yellow box with the required kit in which you can preserve your work for use in the following week.
The boxes should be labelled with your name.

You should maintain a clear record of your work in a laboratory notebook as you work through the checkpoints.
This must include diagrams of the circuits used.
Python code written on the Raspberry Pi must include explanatory comments. %, and at each check point you need to demonstrate that your code compiles and runs correctly.
%Each partner will need to maintain their own notebook to demonstrate that a checkpoint has been completed.

For all checkpoints you will either work individually in the lab, or as part of a pair with one student in the lab and one accessing remotely.
If you are working in pairs you will naturally share results and Python code but you should each provide a separate submission in Learn for the checkpoints, answering any questions independently.
Ensure that you have all data and other results (e.g. photographs of equipment) prepared before you leave the lab.

\section{Projects}

In weeks 7 to 11 of the DAH course you will carry out a project. %during the laboratory sessions. You will continue to attend during the same afternoon as for the checkpoints.
The projects will build upon what you have learned during the checkpoints, but you will also encounter new material.
%While the checkpoints concentrated on specific data acquisition techniques and data handling methods the projects will allow you to progress towards building a small DAQ and/or data analysis system.
While the checkpoints had you collect and analyse small datasets with equipment in the laboratory, for the project you will perform more complex analysis of a large, pre-recorded dataset produced by the LHCb experiment at CERN.
The projects will have an open-ended aspect and are an opportunity where you can show your own initiative and demonstrate your experimental and computational skills. 
You will be able to choose from a list of projects. %, but due to the availability of equipment, the number of spaces for each project will be limited.
%A signup sheet will be provided.

%The equipment specific to each DAH project will be available in red boxes.
%Some of the parts, including Arduinos and loudspeakers, as well as the Raspberry Pi's, will be shared between Tuesday and Thursday afternoon sessions.
%In addition, each pair will continue to use the yellow box in which you can preserve your work for use in the following week. 

%For the DAH project you will work in pairs.
Throughout the project, you should maintain a clear record of your work in your notebook.
%As an example, diagrams of built circuits must be included.
Python code must include explanatory comments.
You will be required to submit a written report for the project, along with your code and results.

\newpage
\section{Assessment} 

Data Acquisition and Handling is a continuously assessed course.
The overall DAH assessment has three parts.
The sum of the marks achieved while carrying out the checkpoints will count for 30\% of the total course mark.
The marks obtained for the DAH project will count for 60\% of the total course mark.
In addition, there will be a quiz/hand-in which will count for 10\% of the total course mark.

\subsection{Assessment of checkpoints}

%There are six checkpoints and checkpoints 2 to 6 will be assessed by one of the demonstrators during the laboratory hours. 
There are five checkpoints, and checkpoints 2 to 5 will be assessed by submission in Learn.
%The assessment will be performed when you decide that you have completed the tasks for a checkpoint or parts thereof as allowed by the marking scheme. For each checkpoint a total score of between 8 and 10 marks will be awarded.  In total up to 45 marks will be available for the checkpoints.
Chapter~\ref{sec:checkpoints} explains what you are expected to submit, and how many marks are available for each section.

Please note the following:
\begin{itemize}
\item In the case that a remote student is paired with a student in the lab, each student will be assessed separately. The marks awarded need not to be the same for both students. 
\item A marking scheme will be provided. Some of the tasks of each checkpoint may be marked together.
\item Be sure to check what is required for submission before you leave the lab.
%\item You will only be awarded marks if you can demonstrate that the relevant circuit functions correctly and that your python code achieves the requested results. 
%\item You will be assessed on your (individual) laboratory notebook and the structure and readability of your (joint) python code.
\item You will be required to answer the questions given in the checkpoints. These questions may require research beyond the materials provided.
%\item You are required to have separate python scripts for each task of a checkpoint. By doing this you will be able to demonstrate all parts of a checkpoint and not just the last task. For some tasks you will need to write results or figures to a file. Make sure that these are not overwritten by the next task.
\item You should be able to complete each checkpoint in one or two lab sessions, and are not required to work on them outside of these hours. Two weeks are allotted for CP2, and you should submit your results by Friday of week 4. Each following checkpoint should be completed in one laboratory session, and results submitted by the Friday of the following week.
\end{itemize}

The overall laboratory assessment will be made from the sum of marks of the check points, which constitutes 30\% of the total course mark.

\subsection{Project assessment}

Your DAH project will be assessed through the submitted material.
This includes your project report, your DAH software (e.g. Python scripts), and any supplementary material you choose to submit.
Guidance on how to prepare these items is given below.
The project will be marked according to the University Common Marking Scheme. 

\newpage
\subsubsection{Report Preparation}

For how to write a proper report we refer you to the workshop slides on report writing in the Senior Honours (SH) Projects course, which are available at \url{https://www.wiki.ed.ac.uk/display/SP/SH+Projects}.
The basic layout of a DAH report will be similar to an SH project report with the main difference being that a good DAH report will be shorter --- typically about 7 pages long.
It is expected that the report is typed.
Most students use LaTeX or Word, either is fine.

When planning and writing a report, you need to be selective about what to include in your report: it should be a concise technical document.
However, it also needs to contain all the information required for the reader to understand what was achieved, i.e. with your report you need to be able to demonstrate to what extent the project was carried out successfully.
%It is often useful to include circuit diagrams, pictures of the setup, or plots of measurements.
A good report would allow a fellow student to be able to reproduce your work.
Students are advised to start writing their report as the project progresses.
Experience shows that report writing always takes longer than you expect!

%When working with a partner, the report must make it clear which parts of the project were carried out together, which parts are only your work, and which parts were only carried out by your partner.
Each student is required to submit an individual report for the project.
The report must contain a signed ``Own Work Declaration,'' which will be available on Learn.

\subsubsection{Programming Code}

Programming code for the DAH project, using Python, written on the Raspberry Pi or on another computer, must include explanatory comments.
When reading a (Python) script, a reader should easily be able to understand what the script will do.
All code written (in Python or another programming language) for the DAH project will need to be submitted using the "Assessment" tool on Learn.
The files should be bundled up in a .zip or .tar file.
A README file should be included.
Submission details will be provided.

\subsubsection{Supplementary Material}

You are encouraged to submit supplementary material if you consider the material as a part of the project that does not fit into the report format. 
This could include your laboratory notebook, output files produced by running a python script, etc.
These can be submitted using the "Assessment" tool on Learn, or as a link to a webpage or other online resource (e.g. Dropbox).
If you have questions about the suitability of material, please consult with the Course Organiser.
All such supplementary material must be clearly listed in an appendix to the report and referred to in the main text. 

\newpage
\subsubsection{Submission Deadline}

The assessed material for the DAH projects will need to be submitted by {\bf 12.00 NOON on Friday, 4th December 2020}. By the deadline you must have submitted 
\begin{itemize}
\item an electronic version of your project report to Turnitin via Learn. At the beginning of your submission you will need to include an ``Own Work Declaration''
\item supplementary material via Learn. %If required (usually not the case), you can submit supplementary material to the Teaching Office in JCMB (Room 4309)
%\item and any supplementary material. 
\end{itemize}
The marks obtained for the DAH project will count for 60\% of the total course mark. 
Reports submitted after the deadline will receive a penalty of 5\% (equivalent to 3 marks out of 60) for each calendar day by which the deadline is exceeded.
Students who, for good reason, find they are unable to meet the deadline, should contact the DAH course administrator before the deadline.

\subsection{Quiz}

Towards the end of the checkpoints, you will be need to submit a quiz/hand-in on questions about data acquisition and handling material.
You will be given two weeks to solve these questions on your own time, i.e. you should not use laboratory hours to solve the quiz questions.
%The exact deadline for handing in the quiz will be announced on Learn, it will be around the end of week 7 of the semester. 
The quiz will count for 10\% of the total course mark, and should be submitted by {\bf 12.00 NOON on Friday, 20th November 2020}.

\section{Plagiarism}

The University regulations on plagiarism apply, see Section 27 of the Taught Assessment Regulations, available online at \url{http://www.ed.ac.uk/schools-departments/academic-services/policies-regulations/regulations/assessment}.
